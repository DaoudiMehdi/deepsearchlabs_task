\documentclass{beamer}
\usetheme{metropolis}
\usepackage{hyperref}

\title{Automated Presentation for Articles by Gemini and El Mehdi}
\date{}

\begin{document}

\maketitle

\begin{frame}
\frametitle{Overview}
\tableofcontents
\end{frame}

\section{climate}

\begin{frame}
\frametitle{Summary 1: Climate Articles}
In this section, we examine several articles focusing on climate change.  We explore the insufficient funding for climate action, the devastating impacts of climate change, and the challenges of achieving international cooperation on climate issues.
\end{frame}

\begin{frame}
\frametitle{Summary 2: Climate Change Impacts and Solutions}
The articles highlight the significant financial challenges in adequately addressing climate change, including the insufficient funding of the Green Climate Fund. They also discuss the disproportionate impact of climate change on developing nations and the urgent need for increased investments to mitigate its effects.  Finally, we examine the political obstacles to decisive climate action.
\end{frame}

\begin{frame}
\frametitle{Main Ideas: Climate}
\begin{itemize}
    \item Insufficient funding for climate action and the Green Climate Fund.
    \item Disproportionate impact of climate change on developing countries.
    \item Political challenges and trade tensions hindering climate action.
    \item The urgent need for increased investments and decisive action.
    \item The importance of mobilizing the private sector for climate investments.
    \item The growing severity of climate disasters and their economic consequences.
\end{itemize}
\end{frame}


\section{companies}

\begin{frame}
\frametitle{Summary 1: Company Performance and Challenges}
This section analyzes various companies' performances, focusing on challenges such as regulatory scrutiny (HSBC), tariff impacts (US food groups, Toniebox), financial struggles (private equity, Neil Woodford), and corporate restructuring (Coca-Cola, Moët Hennessy).
\end{frame}

\begin{frame}
\frametitle{Summary 2: Corporate Strategies and ESG}
We examine corporate responses to political and economic pressures, including the impact of Donald Trump's policies on various industries. We explore different corporate strategies, including asset sales, cost-cutting, and diversification. The role of ESG (environmental, social, and governance) factors is also discussed, with some companies facing criticism for abandoning their sustainability pledges.
\end{frame}

\begin{frame}
\frametitle{Main Ideas: Companies}
\begin{itemize}
    \item Regulatory scrutiny and anti-money laundering concerns impacting HSBC.
    \item The impact of Donald Trump's tariffs on various industries.
    \item Corporate restructuring and strategic decisions in response to economic downturns.
    \item The debate surrounding ESG and companies' commitment to sustainability.
    \item The challenges faced by private equity firms in fundraising and dealmaking.
    \item The evolving landscape of intellectual property rights in the age of AI.
\end{itemize}
\end{frame}


\section{markets}

\begin{frame}
\frametitle{Summary 1: Market Trends and Volatility}
This section examines several key market trends, including the development of a digital euro, the impact of US tariffs on the European car industry, and the increasing popularity of renminbi-denominated bonds. We also analyze the shift in investor sentiment towards emerging markets.
\end{frame}

\begin{frame}
\frametitle{Summary 2: Market Reactions to Global Events}
The articles illustrate market reactions to geopolitical events, economic policy changes, and technological advancements. We explore how investor behavior shifts in response to uncertainty, such as Trump's tariff policies, and the implications for various sectors.  We consider the impact of government intervention on specific markets.
\end{frame}

\begin{frame}
\frametitle{Main Ideas: Markets}
\begin{itemize}
    \item The EU's accelerated plans for a digital euro in response to US stablecoin legislation.
    \item Hedge funds' short positions against European car companies due to tariffs and market slowdown.
    \item The rising popularity of dim sum bonds and the shift away from dollar assets.
    \item Investor sentiment towards emerging markets amid concerns about fiscal dominance in developed economies.
    \item The impact of bank branch closures on access to cash services.
    \item Market reactions to inflation data and the implications for interest rate cuts.
\end{itemize}
\end{frame}


\section{tech}

\begin{frame}
\frametitle{Summary 1: Tech Industry Developments}
This section focuses on various developments within the technology industry, including Sony's gaming strategy, Spotify's price increases, US government investment in Intel, Meta's AI licensing deals, and the impact of AI on content moderation at TikTok.
\end{frame}

\begin{frame}
\frametitle{Summary 2: AI's Impact and Challenges}
We explore the widespread impact of artificial intelligence, from its use in gaming and content moderation to its implications for copyright law and market valuations. We examine both the opportunities and challenges posed by AI, including the ethical considerations and potential for disruption.  The emergence of "agentic AI" is also discussed.
\end{frame}

\begin{frame}
\frametitle{Main Ideas: Tech}
\begin{itemize}
    \item Sony's efforts to balance creativity and control within its gaming empire.
    \item Spotify's strategy of increasing prices while adding new services and features.
    \item US government's increased intervention in the tech sector, exemplified by its investment in Intel.
    \item Meta's decision to license AI technology from third parties.
    \item The impact of AI on content moderation and job displacement at TikTok.
    \item The growing concerns about AI-related valuations and market risks.
    \item The evolution of browser wars in the AI era.
    \item The emergence and challenges of quantum computing.
\end{itemize}
\end{frame}


\section{world}

\begin{frame}
\frametitle{Summary 1: Geopolitical Tensions and Conflicts}
This section covers various geopolitical events, focusing on the tensions between France and the US over antisemitism, Iran's refusal to engage in direct talks with the US, and the ongoing war in Ukraine.  The Israel-Hamas conflict and its implications are also examined.
\end{frame}

\begin{frame}
\frametitle{Summary 2: International Relations and Aid}
The articles discuss international relations, including diplomatic efforts to resolve conflicts and the challenges of providing humanitarian aid. The role of international organizations, such as the UN, is also examined, alongside the impact of funding cuts on various initiatives.
\end{frame}

\begin{frame}
\frametitle{Main Ideas: World}
\begin{itemize}
    \item Diplomatic tensions between France and the US over antisemitism and Israel.
    \item Iran's rejection of direct talks with the US and the internal political pressures.
    \item The ongoing war in Ukraine, Russia's battlefield gains, and the role of Trump's peace push.
    \item The humanitarian crisis in Gaza and the implications of Israel's planned invasion.
    \item The impact of population ageing on economic growth in developed economies.
    \item The need for increased immigration to address labour shortages.
    \item The US Federal Reserve's considerations for interest rate cuts amid inflation and employment concerns.
    \item The potential for the US to source critical minerals from mining waste.
\end{itemize}
\end{frame}

\end{document}